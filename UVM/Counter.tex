\documentclass{article}
\usepackage{graphicx}
\usepackage{hyperref}
\usepackage{listings} % nice code layout
\usepackage[section]{placeins}
\usepackage{xcolor}
\usepackage[utf8]{inputenc}
\lstset{language = Verilog}


\author{Justin Bui}
\title{Simple Adder and Subtractor Testing with UVM}

\begin{document}

\maketitle
\newpage

\tableofcontents
\newpage

\section{Introduction}
This document discusses the adder/subtractor circuit built to help introduce the concepts of UVM. UVM, or the Universay Verification Methodology is a testing and validation platform which allows for the design and development of testbenches and more advanced simulations of our SystemVerilog designs. This project's aim is to briefly introduce the basics of UVM, as well as develop a simple testbench based on the \href{https://www.edaplayground.com/x/296}{UVM Hello World Example}. This project has been completely written and tested using the \bold{EDA Playground}


\section{Project Requirements}
The Memory Display project makes use of implementable BRAM, the 8 digit 7-segment display, switches, and buttons to store, read, and display values. 

\section{Memory Display Implementation}
The current implementation of the Memory Display project builds off of some of the modules built for the previous reaction timer project. The two modules imported are the stopwatch and 7-segment display modules. I use the stopwatch to time the display shift once every 4 seconds. The display is shifted continuously, with 0's being shifted in by default. By pressing the Center button (BTNC), the values selected on the 16 switches are then loaded and displayed on the left-most displays. 

Unfortunately, as of 9/24/2018, I was unable to get the BRAM read/write functionality to work flawlessly, and have decided to presently submit the project in its current state. I will continue to develop the project, working to implement the memory reliably. My plan is to use the Center button as the memory write trigger, recording the switches value into BRAM. I will modify the scroll functionality to automatically load new values from memory (with rollover) when the previous number has scrolled off the display.


\end{document}