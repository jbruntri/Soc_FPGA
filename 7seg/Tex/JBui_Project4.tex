\documentclass{article}
\usepackage{graphicx}
\usepackage[section]{placeins}
\usepackage{xcolor}
\usepackage{hyperref}
\hypersetup{
    colorlinks=true,
    linkcolor=blue,
    filecolor=magenta,      
    urlcolor=cyan,
}
\usepackage[utf8]{inputenc}
\usepackage{listings} % nice code layout
\lstset{language = Verilog}
\lstset{language = C++}
\graphicspath{ {./img/} }
\usepackage[a4paper, margin=1.25in]{geometry}


\author{Justin Bui}
\title{Microblaze Implementation of a 7 Segment Display Controller}

\begin{document}

\maketitle
\newpage

\tableofcontents
\newpage

\section{Introduction}
This document outlines the design and implementation for the 7 Segment Display system. This project makes use of the Digilent NEXYS 4 Artix-7 FPGA development board to implement a Microblaze based 7 Segment display system. The document below includes the functional description, code portions, and implementation of the project. 


\section{Project Description}
The 7 Segment display control system is designed to control the 8 displays present on the Nexys 4 development board. Based on the 32bit Microblaze softcore, this project provides the HDL and drivers necessary to make use of the 7 Segment displays in which ever manner use deems fit. Each display module can be configured individually (that is, independently from one another), allowing for extended flexibility. 

\section{Core Code}
The core functionality of the 7 Segment Display controller is derived from two System Verilog files, the disp\_hex\_mux.sv and bui\_disp\_core.sv. The primary module comprises all 8 of the display modules, as well as control for the decimal points. 

\end{document}